% ****** Start of file aipsamp.tex ******
%
%   This file is part of the AIP files in the AIP distribution for REVTeX 4.
%   Version 4.1 of REVTeX, October 2009
%
%   Copyright (c) 2009 American Institute of Physics.

% Use this file as a source of example code for your aip document.
% Use the file aiptemplate.tex as a template for your document.
\documentclass[%
 aip,
 jmp,%
 amsmath,amssymb,
%preprint,%
 reprint,%
%author-year,%
%author-numerical,%
]{revtex4-1}

\usepackage{graphicx}% Include figure files
\usepackage{grffile}
\usepackage{dcolumn}% Align table columns on decimal point
\usepackage{bm}% bold math
%\usepackage[mathlines]{lineno}% Enable numbering of text and display math
%\linenumbers\relax % Commence numbering lines
\usepackage{multirow}
\usepackage{color} % for the notes
\usepackage{etex}
\reserveinserts{58}
\usepackage{morefloats}
\usepackage{hyperref}
\usepackage{xcolor}
\hypersetup{
        colorlinks,
        linkcolor={red!50!black},
        citecolor={blue!50!black},
        urlcolor={blue!80!black}
}


\maxdeadcycles=1000

\begin{document}

\preprint{XXXXX (preprint)}

%\title[Evolution of interaction networks]{On the evolution of interaction networks: primitive typology of vertex, prominence of measures and activity statistics}% Force line breaks with \\
%\title[Evolution of interaction networks]{On the evolution of interaction networks: a primitive typology of vertex}% Force line breaks with \\
\title[Human interaction typology]{Human interaction networks typology driven from online traces}% Force line breaks with \\

\author{Renato Fabbri}%
 \homepage{http://ifsc.usp.br/~fabbri/}
 \email{fabbri@usp.br}
  \affiliation{ 
S\~ao Carlos Institute of Physics, University of S\~ao Paulo (IFSC/USP)%\\This line break forced with \textbackslash\textbackslash
}

\author{Deborah C. Antunes}
  \homepage{http://lattes.cnpq.br/1065956470701739}
  \email{deborahantunes@gmail.com}
  \altaffiliation{
Curso de Psicologia, Universidade Federal do Cer\'a (UFC)
}%Lines break automatically or can be forced with \\

\author{Marilia M. Pisani}
  \homepage{http://lattes.cnpq.br/6738980149860322}
  \email{marilia.m.pisani@gmail.com}
 \altaffiliation{
Centro de Ciências Naturais e Humanas, Universidade Federal do ABC (CCNH/UFABC)
}%Lines break automatically or can be forced with \\

%
%%\author{Luciano da Fontoura Costa}
%%  \homepage{http://cyvision.ifsc.usp.br/~luciano/}
%%  \email{ldfcosta@gmail.com}
%  \altaffiliation[Also at ]{IFSC-USP}%Lines break automatically or can be forced with \\

%\author{Osvaldo N. Oliveira Jr.}
%  \homepage{www.polimeros.ifsc.usp.br/professors/professor.php?id=4}
%  \email{chu@ifsc.usp.br}
% \altaffiliation[Also at ]{IFSC-USP}%Lines break automatically or can be forced with \\


\date{\today}% It is always \today, today,
             %  but any date may be explicitly specified

\begin{abstract}
	This article reports a human interaction networks typology driven from virtual traces. Such categorization is eased by reports of stability in interaction networks and remarkable differentiation in the textual production of the connective sectors (hubs, intermediary and periphery). The text outlines both a typology of agents and a typology of networks.
\end{abstract}

\pacs{89.75.Fb,05.65.+b,89.65.-s}% PACS, the Physics and Astronomy
\keywords{complex networks, text mining, social network analysis, social psychology, anthropological physics}%Use showkeys class option if keyword
\maketitle

\begin{quotation}
	{\bf == Dummy quotation ==}:
The complexity of representing the words and thoughts of others and relating them to the perspective of ourselves and our interlocutors lies at the heart of our ability to coordinate, distinguish, and calibrate the jostling versions of a partly shared social world. The chapter provides a canonical typology of different types of quotation. There are three canonical types: direct speech, calculated from the primary speech event; indirect speech, calculated from the reported speech event; biperspectival speech, calculated from both perspectives at once. But the number of possibilities between these ideals is immense. Canonical Typology allows us to distinguish a much richer set of possibilities within the large and confusingly labelled set of ‘semi-direct’ and ‘semi-indirect’ phenomena.
\end{quotation}


\section{Introduction}
Human interaction networks.
Typologies.

\subsection{Related work}
What classifications are there about these interaction contexts?
How proximate is current physics and social psychology from the presented framework?

\subsection{Paired articles}
Of the topology, of the textual production and for visualization (all already in arXiv).

\section{Data description}
Use Gmane email messages again? Only Gmane for this first moment?

\section{Methods}
Detection of outliers and common types. Tracing parallels from social psychology and sociology theories.
Feedback of the typology proposals in the studied social networks.

\section{Results and discussion}
To come.

\section{Conclusions and direction for future work}
To come.

Analysis of more data, from different social networks. Tracing complementary typologies and using other classic models.
\begin{acknowledgments}
Renato Fabbri is grateful to CNPq (process: 140860/2013-4,
project 870336/1997-5), United Nations Development Program (PNUD/ONU, contract: 2013/000566; project BRA/12/018)  and 
the Postgraduate Committee of the IFSC/USP. Marília Pisani is ... Deborah Antunes is ...
uthors thank GMANE creators and maintainers. Authors thank referred email lists communities and welcome feedback as core contribution to this, and similar, research.
\end{acknowledgments}


%%%%%%%%%%%%%%%%%%%%%%%%%%%%%%%%%%%%%%%
\appendix
\section{Foo}
Bar

\nocite{*}
\bibliography{paper}% Produces the bibliography via BibTeX.

\end{document}
%
% ****** End of file aipsamp.tex ******



